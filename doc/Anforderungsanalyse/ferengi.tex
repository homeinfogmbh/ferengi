\documentclass[a4paper]{article}

% Paketimporte
\usepackage[ngerman]{babel} % Neue Deutsche Rechtschreibung
\usepackage[utf8]{inputenc} % UTF-8 für plattformunabhängige Codierung mit Umlauten
\DeclareUnicodeCharacter{00A0}{ } % Für no-break-spaces
\usepackage{eurosym} % €-Symbol
\usepackage{titlesec} % Textüberschriften anpassen
\usepackage[pdftex]{graphicx} % Für Grafiken
\usepackage{float} 
\usepackage{url} % Für korrekte URL-Formatierung
\usepackage{wrapfig} % Für Abbildungen mit textumlauf
\usepackage{csquotes} % Für Zitate
\usepackage{glossaries} % Für Glossar

% \titleformat{⟨Überschriftenklasse⟩}[Absatzformatierung⟩]{⟨Textformatierung⟩} {⟨Nummerierung⟩}{⟨Abstand zwischen Nummerierung und Überschriftentext⟩}{⟨Code vor der Überschrift⟩}[⟨Code nach der Überschrift⟩]

\titleformat{\chapter}[hang]{\large\bfseries}{\thechapter\quad}{0pt}{}
\titleformat{\section}[hang]{\large\bfseries}{\thesection\quad}{0pt}{}
\titleformat{\subsection}[hang]{\large\bfseries}{\thesubsection\quad}{0pt}{}
\titleformat{\subsubsection}[hang]{\large\bfseries}{\thesubsubsection\quad}{0pt}{}
\titleformat{\paragraph}[hang]{\large\bfseries}{\theparagraph\quad}{0pt}{}

% \titlespacing{⟨Überschriftenklasse⟩}{⟨Linker Einzug⟩}{⟨Platz oberhalb⟩}{⟨Platz unterhalb⟩}[⟨rechter Einzug⟩]

\titlespacing{\chapter}{0pt}{-3em}{6pt}
\titlespacing{\section}{0pt}{6pt}{6pt}
\titlespacing{\subsection}{0pt}{6pt}{6pt}
\titlespacing{\subsubsection}{0pt}{6pt}{6pt}
\titlespacing{\paragraph}{0pt}{6pt}{6pt}

%Toleranzen für Mikrotypographie
\pretolerance=150
\setlength{\emergencystretch}{10em}

% Metadaten
\title{Anforderungsspezifikation zum Projekt \newline \enquote{Frankly Everything, but Real Estates Notoriously Greedy Importer} (\enquote{FERENGI})}
	\author{Für die HOMEINFO - Digitale Informationssysteme GmbH (\enquote{Kunde}) \\ 
	von Richard Neumann (\enquote{Entwickler})}
\date{\today}

% Inhaltsverzeichnis-Konfiguration
\setcounter{tocdepth}{5}
\setcounter{secnumdepth}{5}


% Glossar
\makeglossaries
	
\begin{document}	
	% Titelblatt
	\maketitle
	\pagebreak
	
	% Inhaltsverzeichnis
	\tableofcontents
	\pagebreak
	
	\section*{Vorwort}
	Diese Spezifikation stellt fest, was entwickelt werden soll. Sie enthält dazu die Anforderungen des Kunden an die Anwendung.
	Vermerke und ausstehende Aktionen sind \emph{kursiv} gedruckt.
	
	\pagebreak
	\section{Mission des Projekts}
	\subsection{Erläuterung des zu lösenden Problems}
	Mit \enquote{FERENGI} soll ein neues zentralisiertes und modulares System zur Beschaffung, Speicherung, Verwaltung, und Filterung von Wetterdaten, Nachrichten, Zitaten und weiteren nicht-Immobilien Zusatzdaten für Digital-Signage Geräte entwickelt werden. 
	\subsection{Wünsche und Prioritäten des Kunden}
	Die \emph{Beschaffung}, \emph{Speicherung} und \emph{Verwaltung} von solchen Zusatzdaten von Drittanbietern steht im Vordergrund.
	Des Weiteren wurde folgende Kundenwünsche definiert:
	\begin{itemize}
		\item Automatisierter Import
		\item Zentrale Speicherung von Daten
		\item Regelmäßige Aktualisierung der Daten
		\item Modularität und Erweiterbarkeit
		\item Einhaltung von Standards
		\item Geringe Komplexität
	\end{itemize}
		
	\subsection{Domäne}
	Die Applikation soll als UNIX-Daemon auf einem Linux-Server laufen und webbasierte Schnittstellen zur Verfügung stellen.
	
	\subsection{Maßnahmen zur Anforderungsanalyse}
	Zur Anforderungsanalyse wurde das aktuelle, auf verteilten PHP Skripten basierende System der Fa. HOMEINFO als Referenz genommen. Weiterhin wurden Anforderungen aus den vom Kunden erbrachten Dienstleistungen hergeleitet.
	Insbesondere durch direkten Kontakt zum Geschäftsführer Herrn Gunkel können laufend Unklarheiten geklärt und Wünsche und Prioritäten besprochen werden.
	
	\begin{tabular}{|p{4cm}|p{7,5cm}|}
		\hline
		\emph{Datum} & \emph{Thema} \\
		\hline
		Mi., 21.08.2014 10:30 Uhr & Erstellung der Anforderungsdefinition \\
		\hline
	\end{tabular}
	
	\section{Rahmenbedingungen und Umfeld}	
	\subsection{Einschränkungen und Vorgaben}
	Der Entwurf ist auf einen UNIX-Daemon beschränkt. Als Schnittstellen zum Benutzer sollen die Protokolle \emph{MySQL} und \emph{SFTP} zum Einsatz kommen. Des Weiteren sind die Schnittstellen zu den Wetterdaten, Nachrichten und Zitaten der Drittanbieter fest vorgegeben.
	
	\subsection{Anwender}
	Das System ist ausschließlich für die Digital-Signange Geräte und Mitarbeiter der Firma \emph{HOMEINFO - Digitale Informationssysteme GmbH} zugänglich.
	Dabei sollen die Exposé-TVs der Firma die aktuellen Immobiliendaten der Kunden automatisiert über asymmetrische Authentifizierung vom System beziehen können.
	\linebreak
	
	\begin{tabular}{|p{3cm}|p{9cm}|}
	\hline
	\emph{Rolle} & \emph{Beschreibung} \\
	\hline
	Benutzer & Kunde der Fa. HOMEINFO. Kann:
		\begin{itemize}
		\item \emph{Keine Aktion}
		\end{itemize} \\
	\hline
	Fa. HOMEINFO & Kunde der Software. Kann:
		\begin{itemize}
		\item Daten von Anbietern automatisch beziehen
		\item Daten von maschinell auf den Exposé-TVs anzeigen lassen
		\end{itemize} \\
	\hline
	Administrator & Eine Person, welche einen Account mit Zugriffsrechten auf alle Systemdaten hat. Kann:
		\begin{itemize}
		\item \emph{Vollzugriff}
		\end{itemize} \\
	\hline
	\end{tabular}
	
	\pagebreak
		
	\subsection{Schnittstellen und angrenzende Systeme}
	\emph{FERENGI} soll sich mit einer Vielzahl von Drittanbietern austauschen können. Dabei soll es von folgenden Systemen Daten aufnehmen können:
	\begin{itemize}
		\item Facebook
		\item Adversign
		\item Wetter.de
	\end{itemize}
	Weitere Schnittstellen sollen jederzeit über eine standardisierte API aufgenommen werden können.
	
%	\section{Funktionale Anforderungen}
%	\emph{FERENGI} soll in der Lage sein, die o.g. Schnittstellen automatisch für jeden Benutzer anzusprechen und die empfangenen Daten lokal im \emph{Openimmo}-Format in der \emph{Version 1.2.6}. zu hinterlegen.
	
%	\subsection{Use-Case-Diagramme}
%	\subsubsection{Use-Case Tabellen}	
%	\begin{tabular}{|p{2,5cm}|p{9cm}|}
%		\hline
%		\emph{Use Case Nr. 1} &  \\
%		\hline
%		\emph{Umfeld} &  \\
%		\hline
%		\emph{Systemgrenzen} &  \\
%		\hline
%		\emph{Ebene} &  \\
%		\hline
%		\emph{Hauptakteur} &  \\
%		\hline
%		\emph{Stakeholder u. Interesse} &  \\
%		\hline
%		\emph{Voraussetzung} &  \\
%		\hline
%		\emph{Garantien} & - \\
%		\hline
%		\emph{Erfolgsfall} & \\
%		\hline
%		\emph{Auslöser} &  \\
%		\hline
%		\emph{Beschreibung} & 
%			\begin{tabular}{lp{8cm}}
%				1. &  \\
%				2. &   \\
%				3. &  \\
%			\end{tabular}  
%		WEITER: Use Case 2 \\
%		\hline
%		\emph{Erweiterungen} & 
%		\begin{tabular}{lp{8cm}}
%			4a & WENN ,  DANN \\
%		\end{tabular} \\
%		\hline
%		\emph{Technologie} &  \\
%		\hline
%
%	\end{tabular}
%	
%	\pagebreak
%	
%	\section{Skizzen}
%	Im Folgenden werden Skizzen zur webbasierten grafischen Benutzeroberfläche gezeigt, die Als Vorlage für die spätere Implementierung dienen sollen.
%	
%	\subsection{Programmabläufe}
%	\begin{figure}[H]
%		\centering
%%		\includegraphics[width=0.7\linewidth]{./GUI/PET-Login}
%		\caption{Expose-Agent Import}
%		\label{fig1}
%	\end{figure}
%	% Glossar ausgeben
%	\printglossaries
	
\end{document}
